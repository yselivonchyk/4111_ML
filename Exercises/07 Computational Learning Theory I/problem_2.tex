Let define an algorithm for learning the concept:
\begin{enumerate}

  \item start from learning concept that contains all possible 3 element conjunctions.
  \item  for each learning example over $\{x_{1},\ldots,x_{n}\}$ we generate all possible 3-element conjunctions that can be deduced from the example.
  \item Remove from the learning concept all the conjunctions that does not correspond to the learning example.

\end{enumerate}


Lets analyse the algorithm.
\[\]
On step 1 learning concept will contain all possible conjunctions. Having 2n possible elements we can find this number N and knowing that after one literal has been used we can not use neither the same literal nor its inverse we can calculate:
$\{x_{1},\ldots,x_{n}\}$
\[
N = 2n*(2n-2)*(2n-4)/3!=(4/3)(n^3-3n^2+2n)
\]

On step 2 for input example we generate next number of conjunctions:
$\{x_{1},\ldots,x_{n}\}$
\[
n*(n-1)*(n-2)/3!=1/6(n^3-3n^2+2n)
\]
Step 3 analyses. Let assume that target learning concept contains total number of $m$ conjunctions in its  structure. Then the probability that a random conjunctions that is within within the current concept but is not included in learning concept is
\[
\frac{(4/3)(n^3-3n^2+2n) - 1/6(n^3-3n^2+2n) - m}{(4/3)(n^3-3n^2+2n) - m}  = 
\frac{7(n^3-3n^2+2n) - 6m}{8(n^3-3n^2+2n) - 6m} =p 
\] 
Where $p$ is a constant number that is polynomially dependent on the size of the example space and the constant characteristic of concept to be learned.

Then the probability to discard an a single wrong conjunctions within $i$ steps is $p^i$. Then using standard limit theorem we can formulate the expected number of disjunctions that we discard after i examples:
\[
E(n) = (N - m)*p^i, \delta(n) = \sqrt{np^i(1-p^i)}
\] 
To show that we find a true concept with requested probability $\delta$ we just need to show that probability of $E(n) - z\delta(n)$ is greater or equal then $1-\delta$. Corresponding value of $p^i$ would always exist because $E(n) - z\delta(n)$ is monotonously increasing function on $p^i \in [0, 1]$ because $(N - m)*p^i$ is always increasing and $\sqrt{np^i(1-p^i)}$ is decreasing.
\[
\]
Therefore, we can always peak proper value of $p^i$ that allows us to find the learning concept with no error with confidence of $\delta$. $p^i$ is a polynomial because $p$ itself is a polynomial. If we can find a solution for $error = 0$ then this solution will be as well true for any $error  \geq 0$. Therefore, without loss of generality we can claim that problem is PAC solvable for any requested values of $error, \delta, n$.
\[
\]
To make the reasoning sound we to remove dependence of function $p$ on hidden variable $m$. We can show, that 
\[
m < (7/8)(4/3)(n^3-3n^2+2n)
\]
because otherwise the target C will contain at least 8 disjunction of the same literals $x_i, x_j, x_k$ and this disjunctions will always yeald "false". In this case target concept can be expressed with any expression that returns "false", for example, any 8 distinct disjunctions of the same literals. Therefore
\[
p \geq \frac{1}{(n^3-3n^2+2n)},
\]
for any learning concept
