\documentclass[a4paper]{article}

% Set exercise number.
\newcommand{\exnumber}{3}

%%% Load base packages.
\usepackage[T1]{fontenc}
\usepackage[latin9]{inputenc}
\usepackage{amsmath}

\usepackage[pdf]{pstricks}
\usepackage{pst-node}
\psset
{
    rowsep=4mm,
    colsep=8mm,
    mnodesize=10mm,
    mnode=r,
}

\newcommand{\C}[1]{[mnode=circle,fillstyle=solid]$#1$}
\newcommand{\R}[1]{[mnode=r]\psframebox[fillstyle=solid]{\parbox{10mm}{\strut\centering$#1$}}}
\def\Symbol{%
\psset{unit=3mm}
\pspicture(-1,-1)(1,1)
    \psset{linewidth=0.4pt}
    \psline(-1,0)(1,0)
    \psline(0,-1)(0,1)
    \psline[linewidth=3\pslinewidth](-0.75,-0.0)(0,-0.0)(0,0.5)(0.75,0.5)
\endpspicture
}


\begin{document}
\thispagestyle{empty} 

\begin{psmatrix}
    $1\:$           & \R{-\frac{1}{2}}\\
    $x_1\:$         & \R{1}\\
    $x_2\:$        	& \R{1}       & \C{\boldsymbol{\Sigma}}    & [mnode=circle,fillstyle=solid,name=symbol]\Symbol   &   $\:f(x)$ \\
    \vdots          & \vdots & & & \text{output} \\
    $x_n\:$         & \R{1}\\
   \text{inputs}   & \text{weights}
    \psset{arrows=->}
    \ncline{1,1}{1,2}
        \ncline{2,1}{2,2}
            \ncline{3,1}{3,2}
                \ncline{5,1}{5,2}
    \ncdiagg{1,2}{3,3}
        \ncdiagg{2,2}{3,3}
            \ncdiagg{3,2}{3,3}
                \ncdiagg{5,2}{3,3}
    \ncline{3,3}{3,4}
    \ncline{3,4}{3,5}
\end{psmatrix}
\end{document}