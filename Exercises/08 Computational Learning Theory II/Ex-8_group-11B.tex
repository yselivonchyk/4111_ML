\documentclass[a4paper]{article}

% Set exercise number.
\newcommand{\exnumber}{8}

%%% Load base packages.
\usepackage[T1]{fontenc}
\usepackage[latin9]{inputenc}
\usepackage{amsmath}
\usepackage{amssymb}
\usepackage{booktabs}
\usepackage{multirow}
\usepackage{fancyhdr}
\usepackage{graphicx}

%%% Add specific functionality.
\usepackage{qtree}		% Tree diagrams.

%%% Adjust the style of the document.

% Seperate paragraphs by vertical space instead of indents.
\setlength{\parskip}{0.6\baselineskip}
\setlength{\parindent}{0em}

% Hanging section numbering
\makeatletter
\def\@seccntformat#1{\llap{\csname the#1\endcsname\quad}}
\makeatother

% Header.
\pagestyle{fancy}	
\fancyhead[EOL]{ILAS 2014/15}
\fancyhead[EOC]{Exercise \exnumber}
\fancyhead[EOR]{Group 11\,B}

% Fix tabular spacing issue.
\setlength\belowcaptionskip{\abovecaptionskip}

\providecommand{\tabularnewline}{\\}


\begin{document}

\title{Intelligent Learning and Analysis Systems: Machine Learning\,---\,Exercise \exnumber}
%\title{ILAS 2014/15\,---\,Exercise \exnumber}	% Alternative title.
\author{Yauhen Selivonchyk\\ {\L}ukasz Segiet\\Duc Duy Pham\\Dominik Schindler}

\maketitle

\section{VC-dimension of some concept classes over $\boldsymbol{\mathbb{R}^2}$.}
\subsection*{Problem a)}
$VC_{dim}(Circles) = 3$\\
\\
\textbf{\underline{\textit{Proof:}}}\\
It is clear that any 2 points can be shattered. For this, just choose any cirlce, s.t. on point is within the circle and the other one outside of it, or a circle that contains both or none of them\\
For 3 non-colinear points, we can also always find a circle, s.t. 2 of then are inside the circle, leaving one out, and vice verca. One simple method is to first choose a circle, s.t. all 3 points lie on the border of the circle. 
To leave one point outside the circle, just move the circle towards the other two points in perpendicular direction to the line going through these two points. To keep this one point inside, move the circle in opposite direction.
The fact that there also always exist circles that contain either none or all of the points is trivial.\\
For 4 points we need to consider 3 cases:
\begin{itemize}
\item The 4 points are colinear:\\
It is clear that we can not find any cirlces, that only contain the first and the third point on the line and exclude the second and the forth.
\item The 4 points form a triangle (convex hull):\\
In this case 3 points are the corner points of the triangle. Here it is also impossible to find a cirlce that contains the corner points and excludes the non corner point.
\item The 4 points form a quadrilateral (convex hull):\\
For each point consider the opposing point with greatest distance. This way we achieve the two diagonals of the quadrilateral.
It is not possible to find a circle that contains the two points forming the longer diagonal while excluding the other two.
\end{itemize}
So we have shown that we can not find any subset of $\mathbb{R}^2$ with magnitude 4, that can be shattered by the concept class of circles.\\
Thus we get $VC_{dim}(Circles) = 3$.
\hfill$\square$


\subsection*{Problem b)}
$VC_{dim}(Triangles) = 7$\\
\\
\textbf{\underline{\textit{Proof:}}}\\
Results directly from Problem c.).
\hfill$\square$

\subsection*{Problem c)}
Let $\mathcal{P}^(k)$ be the class of convex polygons consisting of $k$ vertices.\\
$VC_{dim}(\mathcal{P}^(k)) = 2k+1$\\
\\
\textbf{\underline{\textit{Proof (sketch):}}}\\
Place $d$ points on a circle and label them arbitrarily $+$ or $-$.
We want to find or construct a polygon that contains all points labeled with $+$ und excludes all points labeled with $-$.\\
Let $P$ be the set of positively labeled points and $N$ the set of negatively labeled points.\\
Construct the polygon as follows:\\
If $|P|\leq |N|$ , create a convex polygon with $|P|$ vertices, s.t. the points in $P$ are its vertices. This way every point labeled $+$ is in the polygon and every other point outside of it.\\
If $|P|>|N|$, create a convex polygon with |N| sides, s.t. the sides of the polygon are tangent to the circle in the points of N. This way the circle is within the constructed polycon, so every point labeled $+$ is in the polygon and every point labeled $-$ exactly on one side of the polygon. Now move each side by a very small amount $\epsilon > 0$ towards the center of the circle, s.t. now every point labeled $+$ is still in the polygon but every point labeled $-$ is outside of the polygon.\\
If we have $2k+1$ points, the constructed polygon consists of at most $k$ points. Since we labeled the points arbitrarily, we have found a subset of size $2k+1$ that is shattered by $\mathcal{P}^(k)$.\\
So we can conclude $VC_{dim}(\mathcal{P}^(k)) \geq 2k+1$
We now still need to show that $VC_{dim}(\mathcal{P}^(k)) \neq 2k+2$.
This can be done by firstly showing that placing the points an a circle maximizes the number of possible subsets of this particular subset that can be \textit{realized} by $\mathcal{P}^(k)$. Secondly we would just need to add another point to the circle arbitrarily and find a subset of these $2k+2$ points that can not be \textit{realized} by $\mathcal{P}^(k)$.
\hfill$\square$

\section{Some basic properties of the VC-dimension.}
\subsection*{Problem a)}
Let $\mathcal{C}$ be a concept class with $|\mathcal{C}| < \infty$.\\
Show that $VC_{dim}(\mathcal{C}) \leq \log _2 |\mathcal{C}|$.\\
\\
\textbf{\underline{\textit{Proof:}}}\\
Let $X$ be the instance space and
let $VC_{dim}(\mathcal{C}) = d$ with $d\in \mathbb{N}$.\\
This means that the largest subset $S$ of $X$ that is shattered by $\mathcal{C}$ is of size $d$, i.e. $|S|=d$.\\ 
This again means that any(!!!) subset of  $S$ can be \textit{realized} by some concept in $\mathcal{C}$.\\ 
Since $|S|$ is bounded by $d$, the number of its subsets is bounded by
$|2^{S}|= 2^{|S|} = 2^d$.\\ 
Because for each subset of $S$ there must exist a concept in $\mathcal{C}$ that \textit{realizes} it,\\ 
we can conclude $|\mathcal{C}| \geq |2^{S}|= 2^d$,\\ 
thus $\log _2|\mathcal{C}| \geq d$.
\hfill $\square$

\subsection*{Problem b)}
Let $\mathcal{C}_1, \mathcal{C}_2 \subseteq 2^X$ be concept classes over some instance space $X$.\\
Show that if $\mathcal{C}_1 \subseteq \mathcal{C}_2$ then $VC_{dim}(\mathcal{C}_1) \leq VC_{dim}(\mathcal{C}_2)$.\\
\\
\textbf{\underline{\textit{Proof:}}}\\
Let $VC_{dim}(\mathcal{C}_1) = d_1$ with $d_1\in \mathbb{N}$.\\
Let $S_1$ be the largest subset of $X$ that is shattered by $\mathcal{C}_1$.\\ This means that any subset of  $S_1$ can be \textit{realized} by some concept in $\mathcal{C}_1$.\\ 
Since $\mathcal{C}_1 \subseteq \mathcal{C}_2$, $S_1$ is also shattered by $\mathcal{C}_2$.\\ 
So $VC_{dim}(\mathcal{C}_2)$ is obviously bounded by $|S_1| = VC_{dim}(\mathcal{C}_1)$,\\ 
thus $VC_{dim}(\mathcal{C}_1) \leq VC_{dim}(\mathcal{C}_2)$.
\hfill $\square$

\subsection*{Problem c)}
Let $\mathcal{C} \subseteq 2^X$ be a concept class over $X$ and let $\bar{\mathcal{C}} := \{X \backslash c : c \in \mathcal{C}\}$. Is it
true that $VC_{dim}(\mathcal{C}) = VC_{dim}(\bar{\mathcal{C}})$? Argue why or why not.\\
\\
\textbf{\underline{\textit{Claim:}}}\\
$VC_{dim}(\mathcal{C}) = VC_{dim}(\bar{\mathcal{C}})$\\
\\
\textbf{\underline{\textit{Proof:}}}\\
Let $S$ be a subset of $X$ that is shattered by $\mathcal{C}$.
Let $T \subseteq S$ be any arbitrary subset of $S$.
Since $S$ is shattered by $\mathcal{C}$, we can find a concept in $\mathcal{C}$ that realizes $T$, i.e. $T\in\Pi _{\mathcal{C}}(S)$.\\
\textit{Can we also find a concept in $\mathcal{C}$ that realizes $T$}?\\
To answer this question consider the subset $S\backslash T \subseteq S$ of $S$. Since $S$ is shattered by  $\mathcal{C}$, we can also find a concept $c \in\mathcal{C}$ that realizes $S\backslash T$.
Let $\bar{c}$ be defined as $\bar{c}:= X\backslash c$, which lies in $\bar{\mathcal{C}}$. By definition $\bar{c}$ realizes $X\backslash (S\backslash T)$ w.r.t $X$.
Since $S \subseteq X$, we get $T = S\backslash (S\backslash T) \subseteq (X\backslash (S\backslash T))$. So $\bar{c}$ also realizes $T$ w.r.t. $S$, so $T\in\Pi _{\bar{\mathcal{C}}}(S)$.
Since we have chosen $T\in\Pi _{\mathcal{C}}(S)$ arbitrarilily, we can conclude $\Pi _{\mathcal{C}}(S) \subseteq \Pi  _{\bar{\mathcal{C}}}(S)$.
Thus, any subset of $X$ that is shattered by $\mathcal{C}$ is also shattered by $\bar{\mathcal{C}}$.\\
Apply the same thoughts to $\bar{\mathcal{C}}$ and we get, that any subset of $X$ that is shattered by $\bar{\mathcal{C}}$ is also shattered by $\bar{\bar{\mathcal{C}}} = \mathcal{C}$.\\ 
This leaves us with $VC_{dim}(\mathcal{C}) = VC_{dim}(\bar{\mathcal{C}})$.
\hfill$\square$



\section{Application of the BEHW-Theorem.}
\section{Backpropagation for the XOR Function}

\begin{table}[h]
\begin{centering}
\caption{Learned weights $w_{i,j}$ and run time $T$ (in milliseconds) of
the program for different learning rates $\eta$.\label{tab:table}}

\par\end{centering}

\centering{}%
\begin{tabular}{ccrrrrrrrrr}
\toprule 
\multirow{2}{*}{$\eta$} & \multirow{2}{*}{$T$ (ms)} & \multicolumn{3}{c}{Hidden layer 1 } & \multicolumn{3}{c}{Hidden layer 2} & \multicolumn{3}{c}{Output layer}\tabularnewline
\cmidrule{3-11} 
 &  & $w_{0,1}^{(h)}$ & $w_{1,1}^{(h)}$ & $w_{2,1}^{(h)}$ & $w_{0,2}^{(h)}$ & $w_{1,2}^{(h)}$ & $w_{2,2}^{(h)}$ & $w_{0}^{(o)}$ & $w_{1,2}^{(o)}$ & $w_{2,2}^{(o)}$\tabularnewline
\midrule
$0.1$ & 65 & 2.7 & 2.7 & -1.5 & -1.6 & -1.6 & 2.3 & 2.9 & 3.0 & -2.5\tabularnewline
$0.2$ & 20 & 2.7 & 2.7 & -1.2 & 1.6 & 1.6 & -2.3 & 2.9 & -3.1 & -2.5\tabularnewline
$0.3$ & 17 & 1.7 & 1.7 & -2.4 & 2.6 & 2.6 & -1.1 & -3.0 & 2.9 & -2.5\tabularnewline
$0.4$ & 9 & -1.7 & -1.7 & 2.4 & 2.6 & 2.6 & -1.1 & 3.0 & 2.9 & -2.5\tabularnewline
$0.5$ & 12 & -2.1 & 2.1 & 1.0 & -2.4 & 2.4 & -1.2 & -2.7 & 2.6 & 2.3\tabularnewline
$0.6$ & 8 & -1.7 & -1.7 & 2.4 & -2.6 & -2.5 & 1.1 & 3.0 & -2.8 & -2.5\tabularnewline
$0.7$ & 7 & 1.6 & 1.6 & -2.3 & -2.7 & -2.7 & 1.2 & -3.1 & -2.9 & -2.5\tabularnewline
$0.8$ & 8 & -3.5 & -2.8 & 1.3 & 1.6 & 1.5 & -2.1 & -2.9 & -3.1 & -2.4\tabularnewline
$0.9$ & 5 & -2.1 & 1.0 & -2.0 & -2.5 & 2.5 & 1.2 & 2.7 & -2.6 & 2.3\tabularnewline
$10$ & 6 & 2.7 & 2.7 & -1.1 & 1.8 & 1.8 & -2.6 & 2.8 & -2.8 & -2.4\tabularnewline
\bottomrule
\end{tabular}
\end{table}

The backpropagation algorithm was implemented to learn the XOR function
\[
f(x_{1},x_{2})=(x_{1}\land\lnot x_{2})\lor(\lnot x_{1}\land x_{2}).
\]
The network consist of one hidden layer with two perceptrons and one
output layer with a single perceptron. All perceptrons use the hyperbolic
tangent function for thresholding. We assume $x_{1},\, x_{2}\in\{0,1\}$
and $o\in\{-1,1\}$. For the hidden layer we have
\[
\begin{gathered}z_{1}=\arctan\left(w_{0,1}^{(h)}+w_{1,1}^{(h)}\, x_{1}+w_{2,1}^{(h)}\, x_{2}\right),\\
z_{2}=\arctan\left(w_{0,2}^{(h)}+w_{1,2}^{(h)}\, x_{1}+w_{2,2}^{(h)}\, x_{2}\right),
\end{gathered}
\]
and for the output layer
\[
o=\arctan\left(w_{0}^{(o)}+w_{1,2}^{(o)}\, z_{1}+w_{2,2}^{(o)}\, z_{2}\right).
\]
The program terminates the network training if
\[
\frac{1}{4}\sum_{i=1}^{4}(o(\vec{x}_{i})-t_{i})^{2}<10^{-3}
\]
with 
\[
\begin{gathered}(\vec{x}_{1},t_{1})=\left((0,0),0\right),\\
(\vec{x}_{2},t_{2})=\left((0,1),1\right),\\
(\vec{x}_{3},t_{3})=\left((1,0),1\right),\\
(\vec{x}_{4},t_{4})=\left((1,1),0\right)
\end{gathered}
\]
or if the number of iterations exceeds $10^{6}$. Table \ref{tab:table}
summarizes the results and run time of the program for different learning
rates.

\section{Application of the BEHW-Theorem.}
Let consider the function g(x) which is the compact representation of symmetric boolean function:
\[
g:\{y\} \to \{0, 1\},where \ y\in N, 0 \leq y \leq n
\]
Clearly,
\[
f(\vec x) = g(\sum\limits_{i=1}^n x_i)=g(y)
\]
Any function g(y) can be represented with a table of size equal to n.
Table of size n can perfectly classify any set of inputs of size at most n, therefore we can state that VC(g) <= n. And, as follows, VC(f) <= n.
\[
\]
Now lets introduce binary function h(z).
\[
\vec z \in {0, 1}^j, \ where \ j \in N, j < \sqrt n +1
\]
Where vector z is a binary representation of number k, which corresponds to a number of entry of table representation of function g(y).
\[
\]
Function h(z) can be represented in a CNF with at most $j\leq(\sqrt(n)+1)^2$ underlying disjunctions. Let us introduce conjunctive function i(t), where every element of vector t corresponds to a disjunction of a CNF representation of h(z). 
\[
i(\vec t)| \ t_i \in \{0, 1\}, i \in [0, (\sqrt n+1)^2]
\]
It is already shown, that function i(t) is polynomialy PAC learnable. Learning function i(t) we can reconstruct every presiding function up to f(x) in polynomial time, because every conversion was done in polynomial time.

\end{document}