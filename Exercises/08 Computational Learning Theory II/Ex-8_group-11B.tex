\documentclass[a4paper]{article}

% Set exercise number.
\newcommand{\exnumber}{8}

%%% Load base packages.
\usepackage[T1]{fontenc}
\usepackage[latin9]{inputenc}
\usepackage{amsmath}
\usepackage{amssymb}
\usepackage{booktabs}
\usepackage{multirow}
\usepackage{fancyhdr}
\usepackage{graphicx}

%%% Add specific functionality.
\usepackage{qtree}		% Tree diagrams.

%%% Adjust the style of the document.

% Seperate paragraphs by vertical space instead of indents.
\setlength{\parskip}{0.6\baselineskip}
\setlength{\parindent}{0em}

% Hanging section numbering
\makeatletter
\def\@seccntformat#1{\llap{\csname the#1\endcsname\quad}}
\makeatother

% Header.
\pagestyle{fancy}	
\fancyhead[EOL]{ILAS 2014/15}
\fancyhead[EOC]{Exercise \exnumber}
\fancyhead[EOR]{Group 11\,B}

% Fix tabular spacing issue.
\setlength\belowcaptionskip{\abovecaptionskip}

\providecommand{\tabularnewline}{\\}


\begin{document}

\title{Intelligent Learning and Analysis Systems: Machine Learning\,---\,Exercise \exnumber}
%\title{ILAS 2014/15\,---\,Exercise \exnumber}	% Alternative title.
\author{Yauhen Selivonchyk\\ {\L}ukasz Segiet\\Duc Duy Pham\\Dominik Schindler}

\maketitle

\section{VC-dimension of some concept classes over $\boldsymbol{\mathbb{R}^2}$.}
\subsection*{Problem a)}
\subsection*{Problem b)}
$VC_{dim}(Triangles) = 7$\\
\\
\textbf{\underline{\textit{Proof:}}}\\
Results directly from Problem c.).
\hfill$\square$

\subsection*{Problem c)}
Let $\mathcal{P}^(k)$ be the class of convex polygons consisting of $k$ vertices.\\
$VC_{dim}(\mathcal{P}^{(k)}) = 2k+1$\\
\\
\textbf{\underline{\textit{Proof (sketch):}}}\\
Place $d$ points on a circle and label them arbitrarily $+$ or $-$.
We want to find or construct a polygon that contains all points labeled with $+$ und excludes all points labeled with $-$.\\
Let $P$ be the set of positively labeled points and $N$ the set of negatively labeled points.\\
Construct the polygon as follows:\\
If $|P|\leq |N|$ , create a convex polygon with $|P|$ vertices, s.t. the points in $P$ are its vertices. This way every point labeled $+$ is in the polygon and every other point outside of it.\\
If $|P|>|N|$, create a convex polygon with |N| sides, s.t. the sides of the polygon are tangent to the circle in the points of N. This way the circle is within the constructed polycon, so every point labeled $+$ is in the polygon and every point labeled $-$ exactly on one side of the polygon. Now move each side by a very small amount $\epsilon > 0$ towards the center of the circle, s.t. now every point labeled $+$ is still in the polygon but every point labeled $-$ is outside of the polygon.\\
If we have $2k+1$ points, the constructed polygon consists of at most $k$ points. Since we labeled the points arbitrarily, we have found a subset of size $2k+1$ that is shattered by $\mathcal{P}^{(k)}$.\\
So we can conclude $VC_{dim}(\mathcal{P}^{(k)}) \geq 2k+1$
We now still need to show that $VC_{dim}(\mathcal{P}^{(k)}) \neq 2k+2$.
This can be done by firstly showing that placing the points an a circle maximizes the number of possible subsets of this particular subset that can be \textit{realized} by $\mathcal{P}^{(k)}$. Secondly we would just need to add another point to the circle arbitrarily and find a subset of these $2k+2$ points that can not be \textit{realized} by $\mathcal{P}^{(k)}$.
\hfill$\square$

\section{Some basic properties of the VC-dimension.}
\subsection*{Problem a)}
\subsection*{Problem b)}
Let $\mathcal{C}_1, \mathcal{C}_2 \subseteq 2^X$ be concept classes over some instance space $X$.\\
Show that if $\mathcal{C}_1 \subseteq \mathcal{C}_2$ then $VC_{dim}(\mathcal{C}_1) \leq VC_{dim}(\mathcal{C}_2)$.\\
\\
\textbf{\underline{\textit{Proof:}}}\\
Let $VC_{dim}(\mathcal{C}_1) = d_1$ with $d_1\in \mathbb{N}$.\\
Let $S_1$ be the largest subset of $X$ that is shattered by $\mathcal{C}_1$.\\ This means that any subset of  $S_1$ can be \textit{realized} by some concept in $\mathcal{C}_1$.\\ 
Since $\mathcal{C}_1 \subseteq \mathcal{C}_2$, $S_1$ is also shattered by $\mathcal{C}_2$.\\ 
So $VC_{dim}(\mathcal{C}_2)$ is obviously bounded by $|S_1| = VC_{dim}(\mathcal{C}_1)$,\\ 
thus $VC_{dim}(\mathcal{C}_1) \leq VC_{dim}(\mathcal{C}_2)$.
\hfill $\square$

\subsection*{Problem c)}
Let $\mathcal{C} \subseteq 2^X$ be a concept class over $X$ and let $\bar{\mathcal{C}} := \{X \backslash c : c \in \mathcal{C}\}$. Is it
true that $VC_{dim}(\mathcal{C}) = VC_{dim}(\bar{\mathcal{C}})$? Argue why or why not.\\
\\
\textbf{\underline{\textit{Claim:}}}\\
$VC_{dim}(\mathcal{C}) = VC_{dim}(\bar{\mathcal{C}})$\\
\\
\textbf{\underline{\textit{Proof:}}}\\
Let $S$ be a subset of $X$ that is shattered by $\mathcal{C}$.
Let $T \subseteq S$ be any arbitrary subset of $S$.
Since $S$ is shattered by $\mathcal{C}$, we can find a concept in $\mathcal{C}$ that realizes $T$, i.e. $T\in\Pi _{\mathcal{C}}(S)$.\\
\textit{Can we also find a concept in $\mathcal{C}$ that realizes $T$}?\\
To answer this question consider the subset $S\backslash T \subseteq S$ of $S$. Since $S$ is shattered by  $\mathcal{C}$, we can also find a concept $c \in\mathcal{C}$ that realizes $S\backslash T$.
Let $\bar{c}$ be defined as $\bar{c}:= X\backslash c$, which lies in $\bar{\mathcal{C}}$. By definition $\bar{c}$ realizes $X\backslash (S\backslash T)$ w.r.t $X$.
Since $S \subseteq X$, we get $T = S\backslash (S\backslash T) \subseteq (X\backslash (S\backslash T))$. So $\bar{c}$ also realizes $T$ w.r.t. $S$, so $T\in\Pi _{\bar{\mathcal{C}}}(S)$.
Since we have chosen $T\in\Pi _{\mathcal{C}}(S)$ arbitrarilily, we can conclude $\Pi _{\mathcal{C}}(S) \subseteq \Pi  _{\bar{\mathcal{C}}}(S)$.
Thus, any subset of $X$ that is shattered by $\mathcal{C}$ is also shattered by $\bar{\mathcal{C}}$.\\
Apply the same thoughts to $\bar{\mathcal{C}}$ and we get, that any subset of $X$ that is shattered by $\bar{\mathcal{C}}$ is also shattered by $\bar{\bar{\mathcal{C}}} = \mathcal{C}$.\\ 
This leaves us with $VC_{dim}(\mathcal{C}) = VC_{dim}(\bar{\mathcal{C}})$.
\hfill$\square$



\section{Application of the BEHW-Theorem.}
\paragraph*{Claim}

For any finite concept class $\mathcal{C}$ and for any finite concept
class $\mathcal{H}$, if we draw
\[
m\geq\frac{1}{\epsilon}\left(\ln\left(\vert\mathcal{H}\vert\right)+\ln\left(\frac{1}{\delta}\right)\right)
\]
random examples, and find a consistent hypothesis from $\mathcal{H}$,
then we PAC-learn $\mathcal{C}$ using $\mathcal{H}$.


\paragraph*{Proof}

Let $c\in\mathcal{C}$ be a concept and $E(c,D):X\rightarrow\{0,1\}$
an oracle that return an example from the instance space $X$ under
the probability distribution $D$.

Let $h_{\mathrm{bad}}\in\mathcal{H}$ be a bad hypothesis, i.$\,$e.,
\[
\mathrm{Pr}_{D}(h_{\mathrm{bad}}(x)\neq c(x))>\epsilon.
\]
The probability that $h_{\mathrm{bad}}$ classifies $m$ examples
$\{x_{1},\ldots,x_{m}\}$ returned by $E(c,D)$ correctly is
\[
\mathrm{Pr}_{D}(h_{\mathrm{bad}}(x_{i})=c(x_{i}),\, i\in\{1,\ldots m\})=\left(1-\mathrm{Pr}_{D}(h_{\mathrm{bad}}(x)\neq c(x))\right)^{m}\leq(1-\epsilon)^{m}.
\]
The probability that there exists a bad hypothesis in $\mathcal{H}$
that classifies $\{x_{1},\ldots,x_{m}\}$ correctly is bounded by
\[
\mathrm{Pr}_{D}(\exists\; h_{\mathrm{bad}}\in\mathcal{H})\leq\vert\mathcal{H}\vert\,(1-\epsilon)^{m}.
\]
If we apply an upper bound $\delta$, we can solve for $m$.

\[
\begin{aligned} & \vert\mathcal{H}\vert\,(1-\epsilon)^{m}\leq\delta\\
\Rightarrow\; & \vert\mathcal{H}\vert\,(\mathrm{e}^{-\epsilon})^{m}\leq\delta\\
\Leftrightarrow\; & m\,(-\epsilon)\leq\ln\left(\frac{\delta}{\vert\mathcal{H}\vert}\right)\\
\Leftrightarrow\; & m\geq\frac{1}{\epsilon}\left(\ln(\vert\mathcal{H}\vert)+\ln\left(\frac{1}{\delta}\right)\right)\quad\blacksquare
\end{aligned}
\]

\section{Application of the BEHW-Theorem.}
\section{Distances}
\begin{itemize}
\item \(l_1\) norm is defined as: \(l_1(x,y) = \sum_i |x_i-y_i|\) for \(x,y \in \mathbb R ^m\). We can use this norm on the subset \(\{0,1\}^m \subset \mathbb R ^m\), call this induced norm \(D_{\mathrm{Hamming}}\).
\\
Claim: \(D_{\mathrm{Hamming}}\) is a metric.
\\
Proof: Assume \(x,y,z \in \{0,1\}^m\), then:
\begin{itemize}
\item \(l_1(x,y) = \sum_i |x_i-y_i| \geq 0\) and \(l_1(x,y) = 0 \iff x_i = y_i\ \forall i \iff x= y\) \checkmark\(_{\mathrm{non-negativity}}\)
\item \(l_1(x,y) = \sum_i |x_i-y_i| = l_1(y,x) \) \checkmark\(_{\mathrm{symmetry}}\)
\item \(l_1(x,y)+l_1(y,z) = \sum_i |x_i-y_i| + \sum_i |y_i-z_i| \geq \sum_i |x_i-z_i| = l_1(x,z)\) \checkmark\(_{\mathrm{\triangle -ineq.}}\) \(\blacksquare\)
\end{itemize}
\item \(U\) is a finite set and \(D_\Delta : 2^U \times 2^U \to \mathbb N_0\), s.t.
\[ D_\Delta(S_1,S_2) = |S_1 \Delta S_2| \]
for all \(S_1,S_2 \subseteq U\).
\\
Claim: \(D_\Delta\) is a metric on \(2^U\).
\\
Proof: Assume \(S_1,S_2,S_3 \subseteq U\), then:
\begin{itemize}
\item \(D_\Delta(S_1,S_2) =  |S_1 \Delta S_2| \geq 0\) and \(D_\Delta(S_1,S_2) = 0 \iff S_1 \Delta S_2 = \emptyset \iff S_1 = S_2\) (by definition of \(\Delta\)) \checkmark\(_{\mathrm{non-negativity}}\)
\item \(D_\Delta(S_1,S_2) =  |S_1 \Delta S_2| = |S_2 \Delta S_1|  = D_\Delta(S_2,S_1)\) \checkmark\(_{\mathrm{symmetry}}\)
\item First notice that \(x\in S_1 \Delta S_3 \implies x\in S_1 \Delta S_2\) or \(x \in S_3 \Delta S_2\) (w.l.o.g. we may assume that \(x\in S_1\) and \(x\not\in S_3\), then if \(x\not\in S_2\), we have \(x\in S_1 \Delta S_2\) and if \(x \in S_2\), then \(x\in S_3 \Delta S_2\)).
\\
Hence we have:
\[ D_\Delta(S_1,S_2) + D_\Delta(S_2,S_3) =  |S_1 \Delta S_2| +  |S_2 \Delta S_3| \geq  |S_1 \Delta S_3| =
 D_\Delta(S_1,S_3) \]
\checkmark\(_{\mathrm{\triangle -ineq.}}\) \(\blacksquare\)



\end{itemize}


\end{itemize}

\end{document}