\documentclass[a4paper]{article}

% Set exercise number.
\newcommand{\exnumber}{5}

%%% Load base packages.
\usepackage[T1]{fontenc}
\usepackage[latin9]{inputenc}
\usepackage{amsmath}
\usepackage{amssymb}
\usepackage{booktabs}
\usepackage{multirow}
\usepackage{fancyhdr}
\usepackage{graphicx}

%%% Add specific functionality.
\usepackage{qtree}		% Tree diagrams.

%%% Adjust the style of the document.

% Seperate paragraphs by vertical space instead of indents.
\setlength{\parskip}{0.6\baselineskip}
\setlength{\parindent}{0em}

% Hanging section numbering
\makeatletter
\def\@seccntformat#1{\llap{\csname the#1\endcsname\quad}}
\makeatother

% Header.
\pagestyle{fancy}	
\fancyhead[EOL]{ILAS 2014/15}
\fancyhead[EOC]{Exercise \exnumber}
\fancyhead[EOR]{Group 11}


\begin{document}

\title{Intelligent Learning and Analysis Systems: Machine Learning\,---\,Exercise \exnumber}
%\title{ILAS 2014/15\,---\,Exercise \exnumber}	% Alternative title.
\author{Yauhen Selivonchyk\\ {\L}ukasz Segiet\\Duc Duy Pham\\Dominik Schindler}

\maketitle

\section{Voronoi diagrams}
\subsection*{Problem (a)}

Because total number of independent experiments exceeds 50 and number of both positive and negative outputs is at least 10 we are allowed to use central limit theorem and say that we have normal distribution. Lets use if for calculating the confidence interval.

Lets calculate probability of committing an error and probability of correctly classifying an example within the given set:
\[
p=\frac{65-10}{65}=0.846
\]
For given number of experiments standard deviation can be calculated as follows:
 \[
\sigma=\sqrt{n\cdot p\,(1-p)}=\sqrt{65\cdot 0.846(1-0.846)}=2.91
\]

We can look up required z-value for one-sided interval that will cover at least 95\% of the area under normal curve:
 \[
z_1=1.65,\; P(x\in Z_1)=0.9505
\]
One-sided confidence interval in our case will look as follows:
 \[
x\in [0, \bar{x}+z_1\cdot\sigma],\;
x\in [0, 14.8]
\]
Therefore, we can say that true error is upper-bounded by value 15 with probability at least 95\%.

\subsection*{Problem (b)}

For 90\% one-sided confidence interval we should use different value of z:
 \[
z_2=1.19,\; P(x\in Z_2)=0.9015
\]
One-sided confidence interval in our case will look as follows:
 \[
x\in [0, \bar{x}+z_2\cdot\sigma],\;
x\in [0, 13.46]
\]
 \[
e < 13.46/65
\]
Therefore, we can say that true error is upper-bounded by value 0.184 with probability at least 90\%.


\section{{\itshape k}\hspace{0.4ex}-NN}
\subsection*{Problem (a)}

Let the example $x$ from the Training set $S$ have boolean attributes
$B_{x,i},\, i\in\{1,\ldots,n_{B}\}$, categorical attributes $C_{x,j},\, j\in\{1,\ldots,n_{C}\}$
and numerical attributes $N_{x,k},\, k\in\{1,\ldots,n_{N}\}$. Assume
that the categorical attribute $C_{j}$ contains $m_{j}$ different
attributes and that the the numerical attribute $N_{k}$ has standard
deviation $\sigma_{k}$ in the training set $S$. Then we define for
two examples $x$ and $y$
\begin{align*}
\beta_{i}(x,y) & =\begin{cases}
1 & \text{if}\; B_{x,i}=B_{y,i}\\
0 & \text{otherwise}
\end{cases}\\
\gamma_{j}(x,y) & =\begin{cases}
\frac{m_{j}}{2} & \text{if}\; C_{x,j}=C_{y,j}\\
0 & \text{otherwise}
\end{cases}\\
\eta_{k}(x,y) & =\sigma_{k}\,\vert N_{x,k},N_{y,k}\vert
\end{align*}
From these quantities a metric $\delta$ is derived.
\[
\delta(x,y)=\sum_{i=1}^{n_{B}}\beta_{i}(x,y)+\sum_{j=1}^{n_{C}}\gamma_{j}(x,y)+\sum_{k=1}^{n_{N}}\eta_{k}(x,y)
\]
\subsection*{Problem (b)}
\subsection*{Problem (c)}

\section{Hypothesis Testing}
\subsection*{Problem (a)}

Out of $n=100$ examples, $r=87$ are correctly classified. By the
given assumptions, the number of correctly classified examples $X$
is binomial distributed with unknown parameter $p$.
\[
\mathbb{P}(X=k)=\frac{n!}{k!(n-k)!}p^{k}(1-p)^{n-k}.
\]
The binomial distribution has mean
\[
\mathbb{E}(X)=\sum_{k=0}^{n}k\,\mathbb{P}(X=k)=np
\]
and standard deviation 
\[
\sigma_{X}=\sqrt{np\,(1-p)}.
\]
This motivates to estimate $p$ as $\hat{p}=r/n=0.87$ and $\sigma_{X}$%
\footnote{I have also seen the formula with a correction factor of $n/(n-1)$.
Does anyone know more about that?%
} as 
\[
\sigma_{X}\approx\sqrt{n\hat{p}\,(1-\hat{p})}.
\]
 The standard deviation of estimate $\hat{p}$ (this is a random variable)
is
\[
\sigma_{\hat{p}}=\frac{\sigma_{X}}{n}\approx\sqrt{\frac{\hat{p}\,(1-\hat{p})}{n}}=0.0336.
\]
Since $n\geq5$, the binomial distribution can be approximated with
a Gaussian distribution. Therefore the distribution for $\hat{p}=r/n$
is also well approximated by a Gaussian distribution. We have
\[
\hat{p}\sim\mathcal{N}(p,\sigma_{\hat{p}}).
\]
The boundaries for the shortest interval, that contains $\hat{p}$
with $95\,\%$ probability, are found with the table.
\[
\mathbb{P}\left(\hat{p}\in\left[p-1.96\,\sigma_{X},p+1.96\,\sigma_{X}\right]\right)=0.95
\]
We find the confidence interval by 
\begin{align*}
\mathbb{P}\left(\hat{p}\in\left[p-1.96\,\sigma_{X},p+1.96\,\sigma_{X}\right]\right) & =\mathbb{P}\left(\vert p-\hat{p}\vert\leq1.96\,\sigma_{X}\right)\\
 & =\mathbb{P}\left(\vert\hat{p}-p\vert\leq1.96\,\sigma_{X}\right)\\
 & =\mathbb{P}\left(p\in\left[\hat{p}-1.96\,\sigma_{X},\hat{p}+1.96\,\sigma_{X}\right]\right)\\
 & =0.95
\end{align*}
Thus, the $95\,\%$ confidence interval is $\left[\hat{p}-1.96\,\sigma_{X},\hat{p}+1.96\,\sigma_{X}\right]\approx\left[0.804,0.936\right]$.
\subsection*{Problem (b)}

\end{document}